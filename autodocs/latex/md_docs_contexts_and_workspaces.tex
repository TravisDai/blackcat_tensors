Black\+Cat\+\_\+\+Tensors\textquotesingle{}s support using Cuda streams. A stream stores the required meta-\/data of handling computations of a Tensor.

In regards to Cuda, a stream stores a unique cublas\+Handle\+\_\+t, cuda\+Stream\+\_\+t, cuda\+Event\+\_\+t, and a memory\+\_\+pool.

The cpu version contains a Host\+Stream object and a memory\+\_\+pool.

A stream by default uses the default stream. (The host\+Stream, behaves similar)

Tensors using non-\/default streams have their operations run asynchrously to the host thread.

The stream used on expression is based upon the assignment operator of a single expression.


\begin{DoxyCode}
\hyperlink{classBC_1_1tensors_1_1Tensor__Base}{BC::Matrix<float>} y, w, x;

y.create\_stream();
w.create\_stream();
x.create\_stream();

y = w * x; \textcolor{comment}{//will use Y's Stream and stream }
\end{DoxyCode}


Streams will be propagated to slices of a tensor.


\begin{DoxyCode}
y.get\_stream() == y[0].get\_stream(); \textcolor{comment}{//True }
\end{DoxyCode}



\begin{DoxyCode}
\textcolor{keyword}{using} \hyperlink{namespaceBC_abc64a63cd29a22d102a68f478dfd588d}{Stream} = BC::streams::implementation<system\_tag>; 
\hyperlink{namespaceBC_abc64a63cd29a22d102a68f478dfd588d}{Stream} stream;

stream.create\_stream();         \textcolor{comment}{//calls cudaCreateStream}
stream.set\_stream(cudaStream\_t or HostStream)  \textcolor{comment}{//sets the stream }
\end{DoxyCode}
 
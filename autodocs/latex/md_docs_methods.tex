{\ttfamily value\+\_\+type} is used to denote the scalar-\/type of the tensor. {\ttfamily allocator\+\_\+t} is used to denote the allocator-\/template argument. {\ttfamily expression\+\_\+t} is used to represent any non-\/evaluated mathematical expression. Note\+: Many of the return types have been abreviated. The underlying implementation of these is not relevant to the user.

\paragraph*{Data Access}

\tabulinesep=1mm
\begin{longtabu} spread 0pt [c]{*7{|X[-1]}|}
\hline
\rowcolor{\tableheadbgcolor}{\bf static }&{\bf return type }&{\bf method name }&{\bf parameters }&{\bf const/non-\/const }&{\bf documentation }&{\bf alias-\/methods  }\\\cline{1-7}
\endfirsthead
\hline
\endfoot
\hline
\rowcolor{\tableheadbgcolor}{\bf static }&{\bf return type }&{\bf method name }&{\bf parameters }&{\bf const/non-\/const }&{\bf documentation }&{\bf alias-\/methods  }\\\cline{1-7}
\endhead
--- &slice &operator\mbox{[}\mbox{]} &\hyperlink{namespaceBC_a6007cbc4eeec401a037b558910a56173}{B\+C\+::size\+\_\+t} &both &Returns a slice of the tensor. IE Cube returns a matrix slice, Matrix returns a column, etc &slice \\\cline{1-7}
--- &scalar\+\_\+obj &operator() &\hyperlink{namespaceBC_a6007cbc4eeec401a037b558910a56173}{B\+C\+::size\+\_\+t} &both &Returns a scalar object. Access to this data is safe. &scalar \\\cline{1-7}
\end{longtabu}
$\vert$ --- $\vert$ vector $\vert$ diag $\vert$ int=0 $\vert$ both $\vert$ Returns the diagnol of a matrix. A positive integer will have the diagnol start from the top left corner. A negative integer will have the diagnol end n from the bottom right $\vert$ $\vert$ --- $\vert$ slice $\vert$ col $\vert$ \hyperlink{namespaceBC_a6007cbc4eeec401a037b558910a56173}{B\+C\+::size\+\_\+t} $\vert$ both $\vert$ Returns a column of a matrix. $\vert$ $\vert$ --- $\vert$ transpose\+\_\+view $\vert$ transpose $\vert$ --- $\vert$ both $\vert$ returns a transpose view of a Matrix or Vector. Cannot transpose in place. Matrix = Matrix.\+transpose() is undefined. $\vert$ t $\vert$ $\vert$ --- $\vert$ vector $\vert$ row $\vert$ \hyperlink{namespaceBC_a6007cbc4eeec401a037b558910a56173}{B\+C\+::size\+\_\+t} $\vert$ both $\vert$ returns a row of a matrix. $\vert$ $\vert$ --- $\vert$ view $\vert$ transpose $\vert$ \hyperlink{namespaceBC_a6007cbc4eeec401a037b558910a56173}{B\+C\+::size\+\_\+t} $\vert$ both $\vert$ returns a row of a matrix. $\vert$ $\vert$ X $\vert$ reshape $\vert$ reshape $\vert$ tensor, and ints... $\vert$ both $\vert$ returns a reshaped view of the tensor parameter. Does not modify the original tensor $\vert$ $\vert$ X $\vert$ chunk $\vert$ chunk $\vert$ tensor, and ints... $\vert$ both $\vert$ returns a reshaped view of the tensor parameter. Does not modify the original tensor $\vert$

\paragraph*{Iterator Methods}

\tabulinesep=1mm
\begin{longtabu} spread 0pt [c]{*7{|X[-1]}|}
\hline
\rowcolor{\tableheadbgcolor}{\bf static }&{\bf return type }&{\bf method name }&{\bf parameters }&{\bf const/non-\/const }&{\bf documentation }&{\bf alias-\/methods  }\\\cline{1-7}
\endfirsthead
\hline
\endfoot
\hline
\rowcolor{\tableheadbgcolor}{\bf static }&{\bf return type }&{\bf method name }&{\bf parameters }&{\bf const/non-\/const }&{\bf documentation }&{\bf alias-\/methods  }\\\cline{1-7}
\endhead
--- &cw\+\_\+iterator &begin &--- &both &Returns the begining of a coefficientwise iterator. &--- \\\cline{1-7}
--- &cw\+\_\+iterator &end &--- &both &Returns the end of a coefficientwise iterator. &--- \\\cline{1-7}
--- &cw\+\_\+iterator &cbegin &--- &const &Explcit const version of begin. &--- \\\cline{1-7}
--- &cw\+\_\+iterator &cend &--- &const &Explcit const version of end. &--- \\\cline{1-7}
--- &cw\+\_\+iterator &rbegin &--- &both &Returns the begining of a coefficientwise reverse iterator. &--- \\\cline{1-7}
--- &cw\+\_\+iterator &rend &--- &both &Returns the begining of a coefficientwise reverse iterator. &--- \\\cline{1-7}
--- &cw\+\_\+iterator &crbegin &--- &const &Explicit const version of rbegin. &--- \\\cline{1-7}
--- &cw\+\_\+iterator &crend &--- &const &Explicit const version of rend. &--- \\\cline{1-7}
--- &nd\+\_\+iterator &nd\+\_\+begin &--- &both &Returns the begining of a multidimensional iterator (iterates along outer stride). &--- \\\cline{1-7}
--- &nd\+\_\+iterator &nd\+\_\+end &--- &both &Returns the end of a multidimensional iterator. &--- \\\cline{1-7}
--- &nd\+\_\+iterator &nd\+\_\+cbegin &--- &const &Explicit const version of nd\+\_\+begin. &--- \\\cline{1-7}
--- &nd\+\_\+iterator &nd\+\_\+cend &--- &const &Explicit const version of nd\+\_\+end. &--- \\\cline{1-7}
--- &nd\+\_\+iterator &nd\+\_\+rbegin &--- &both &Returns the begining of a multidimensional reverse iterator. (Iterates along outer stride). &--- \\\cline{1-7}
--- &nd\+\_\+iterator &nd\+\_\+rend &--- &both &Returns the end of a multidimensional reverse iterator. &--- \\\cline{1-7}
--- &nd\+\_\+iterator &nd\+\_\+crbegin &--- &const &Explicit const version of nd\+\_\+rbegin. &--- \\\cline{1-7}
--- &nd\+\_\+iterator &nd\+\_\+crend &--- &const &Explicit const version of nd\+\_\+rend. &--- \\\cline{1-7}
--- &cw\+\_\+iterator &iter &int=0, int=size &both &Returns an iterator proxy, used for range-\/convienance. &--- \\\cline{1-7}
--- &cw\+\_\+iterator &nd\+\_\+iter &int=0, int=size &both &Returns an iterator proxy, used for range-\/convienance. &--- \\\cline{1-7}
--- &cw\+\_\+iterator &reverse\+\_\+iter &int=0, int=size &both &Returns an iterator proxy, used for range-\/convienance. &--- \\\cline{1-7}
--- &cw\+\_\+iterator &nd\+\_\+reverse\+\_\+iter &int=0, int=size &both &Returns an iterator proxy, used for range-\/convienance. &--- \\\cline{1-7}
\end{longtabu}
\paragraph*{Operations}

\tabulinesep=1mm
\begin{longtabu} spread 0pt [c]{*7{|X[-1]}|}
\hline
\rowcolor{\tableheadbgcolor}{\bf static }&{\bf return type }&{\bf method name }&{\bf parameters }&{\bf const/non-\/const }&{\bf documentation }&{\bf alias-\/methods  }\\\cline{1-7}
\endfirsthead
\hline
\endfoot
\hline
\rowcolor{\tableheadbgcolor}{\bf static }&{\bf return type }&{\bf method name }&{\bf parameters }&{\bf const/non-\/const }&{\bf documentation }&{\bf alias-\/methods  }\\\cline{1-7}
\endhead
--- &tensor\& &+= &tensor or scalar &non-\/const &--- &--- \\\cline{1-7}
--- &tensor\& &-\/= &tensor or scalar &non-\/const &--- &--- \\\cline{1-7}
--- &tensor\& &\%= &tensor or scalar &non-\/const &Element-\/wise multiplication (to differentiate from matrix multiplication) &--- \\\cline{1-7}
--- &tensor\& &/= &tensor or scalar &non-\/const &--- &--- \\\cline{1-7}
--- &expression\+\_\+t &+ &tensor or scalar &both &--- &--- \\\cline{1-7}
--- &expression\+\_\+t &-\/ &tensor or scalar &both &--- &--- \\\cline{1-7}
--- &expression\+\_\+t &\% &tensor or scalar &both &Element-\/wise multiplication (to differentiate from matrix multiplication) &--- \\\cline{1-7}
--- &expression\+\_\+t &/ &tensor or scalar &both &--- &--- \\\cline{1-7}
--- &expression\+\_\+t &== &tensor or scalar &both &--- &--- \\\cline{1-7}
--- &expression\+\_\+t &$>$ &tensor or scalar &both &--- &--- \\\cline{1-7}
--- &expression\+\_\+t &$<$ &tensor or scalar &both &--- &--- \\\cline{1-7}
--- &expression\+\_\+t &$>$= &tensor or scalar &both &--- &--- \\\cline{1-7}
--- &expression\+\_\+t &$<$= &tensor or scalar &both &--- &--- \\\cline{1-7}
--- &expression\+\_\+t &$\ast$ &tensor or scalar &both &Executes one of the following B\+L\+AS calls gemm, gemv, ger, dot, or scalarmul depending upon the dimensionality of the parameters. This is detected at compile-\/time, and does not incur any branching &--- \\\cline{1-7}
--- &expression\+\_\+t &-\/ &--- &both &Negation of a tensor. &--- \\\cline{1-7}
\end{longtabu}
$\vert$ --- $\vert$ expression\+\_\+t $\vert$ un\+\_\+expr $\vert$ functor $\vert$ Returns a user-\/defined unary\+\_\+expression object that will be laziliy evaluated. $\vert$ --- $\vert$ $\vert$ --- $\vert$ expression\+\_\+t $\vert$ bi\+\_\+expr $\vert$ functor, tensor or scalar $\vert$ Returns a user-\/defined binary\+\_\+expression object that will be laziliy evaluated. $\vert$ --- $\vert$

\paragraph*{Functions}

\tabulinesep=1mm
\begin{longtabu} spread 0pt [c]{*7{|X[-1]}|}
\hline
\rowcolor{\tableheadbgcolor}{\bf static }&{\bf return type }&{\bf method name }&{\bf parameters }&{\bf const/non-\/const }&{\bf documentation }&{\bf alias-\/methods  }\\\cline{1-7}
\endfirsthead
\hline
\endfoot
\hline
\rowcolor{\tableheadbgcolor}{\bf static }&{\bf return type }&{\bf method name }&{\bf parameters }&{\bf const/non-\/const }&{\bf documentation }&{\bf alias-\/methods  }\\\cline{1-7}
\endhead
--- &value\+\_\+type &min &--- &both &--- &--- \\\cline{1-7}
--- &value\+\_\+type &max &--- &both &--- &--- \\\cline{1-7}
--- &void &rand &--- &--- &non-\/const &randomize \\\cline{1-7}
--- &functor &for\+\_\+each &functor &both &Convenient-\/definition of for\+\_\+each. Identical to B\+C\+::for\+\_\+each(tensor.\+begin(), tensor.\+end(), functor) &--- \\\cline{1-7}
--- &void &sort &--- &non-\/const &Implemenation is dependent upon gpu vs cpu allocation and std and thrust\textquotesingle{}s implementation. &--- \\\cline{1-7}
\end{longtabu}


\paragraph*{Utility}

\tabulinesep=1mm
\begin{longtabu} spread 0pt [c]{*7{|X[-1]}|}
\hline
\rowcolor{\tableheadbgcolor}{\bf static }&{\bf return type }&{\bf method name }&{\bf parameters }&{\bf const/non-\/const }&{\bf documentation }&{\bf alias-\/methods  }\\\cline{1-7}
\endfirsthead
\hline
\endfoot
\hline
\rowcolor{\tableheadbgcolor}{\bf static }&{\bf return type }&{\bf method name }&{\bf parameters }&{\bf const/non-\/const }&{\bf documentation }&{\bf alias-\/methods  }\\\cline{1-7}
\endhead
--- &void &print &--- &both &Formatted print to console. &--- \\\cline{1-7}
--- &void &print\+\_\+sparse &--- &both &Formatted print to console, ignoring 0\textquotesingle{}s. &--- \\\cline{1-7}
--- &void &print\+\_\+dimensions &--- &both &Output dimensions of a tensor. &--- \\\cline{1-7}
--- &void &print\+\_\+leading\+\_\+dimensions &--- &both &Output outer dimensions of a tensor (strides). &--- \\\cline{1-7}
--- &void &print\+\_\+block\+\_\+dimensions &--- &both &Output the block\+\_\+dimensions of a tensor (IE a 3x4 matrix will output. {\ttfamily \mbox{[}3\mbox{]}\mbox{[}12\mbox{]}}) &--- \\\cline{1-7}
\end{longtabu}
\paragraph*{C\+Math}

The following Cmath functions are supported through the {\ttfamily \hyperlink{namespaceBC}{BC}} namespace. These expressions will automatically be scalarized. (lazy evaluated) 
\begin{DoxyCode}
1 abs
2 acos
3 acosh
4 sin
5 asin
6 asinh
7 atan
8 atanh
9 cbrt
10 ceil
11 cos
12 cosh
13 exp
14 exp2
15 fabs
16 floor
17 fma
18 isinf
19 isnan
20 log
21 log2
22 lrint
23 lround
24 modf
25 sqrt
26 tan
27 tanh
\end{DoxyCode}
 \#\#\#\# Other functions 
\begin{DoxyCode}
1 logistic              1 / (1 + exp(-x))
2 dx\_logistic,          x * (1 - logistic(x))
3 cached\_dx\_logistic    x * (1 - x;
4 dx\_tanh               1 - pow(tanh(x, 2))
5 cached\_dx\_tanh        1 - pow(x, 2)
6 relu                  max(0, x)
7 dx\_relu               x > 0 ? 1 : 0
8 cached\_dx\_relu        x > 0 ? 1 : 0
\end{DoxyCode}
 \paragraph*{Algorithms}

Algorithms are not implemented by B\+CT. They are forwarded to the standard library or thrust implementation. The purpose of B\+C\+::\+Algorithms is to automatically forward to the correct implementation by detecting how the memory of a tensor was allocated at compile time. 
\begin{DoxyCode}
1 for\_each
2 count
3 count\_if
4 find
5 find\_if
6 find\_if\_not
7 copy
8 copy\_if
9 copy\_n
10 fill
11 fill\_n
12 transform
13 generate
14 generate\_n
15 replace
16 replace\_if
17 replace\_copy
18 replace\_copy\_if
19 swap
20 swap\_ranges
21 reverse
22 reverse\_copy
23 stable\_sort
24 max\_element
25 min\_element
26 minmax\_element
27 equal
\end{DoxyCode}
 